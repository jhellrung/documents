\documentclass{article}

\usepackage{fullpage}

\usepackage{amsmath}
\usepackage{amssymb}
\usepackage{amsthm}

\providecommand{\abs}[1]{\left\lvert#1\right\rvert}
\providecommand{\norm}[1]{\left\lVert#1\right\rVert}

\begin{document}

\title{Math 269B, 2012 Winter, Homework 3 (Solutions)}
\date{February 17, 2012}
\author{Professor Joseph Teran \and Jeffrey Lee Hellrung, Jr.}
\maketitle

\section{Theory}

\begin{itemize}

\item[1.] (Strikwerda 5.1.2.) Show that the modified leapfrog scheme (5.1.6) is stable for $\epsilon$ satisfying
\begin{equation*}
0 < \epsilon \leq 1 \quad \text{if} \quad 0 < a^2 \lambda^2 \leq \frac{1}{2}
\end{equation*}
and
\begin{equation*}
0 < \epsilon \leq 4 a^2 \lambda^2 \left( 1 - a^2 \lambda^2 \right) \quad \text{if} \quad \frac{1}{2} \leq a^2 \lambda^2 < 1.
\end{equation*}
Note that these limits are not sharp. It is possible to choose $\epsilon$ larger than these limits and still have the scheme be stable.

\textbf{Solution}

Continuing from the text, we find the amplification factors to be
\begin{equation*}
g_{\pm}(\theta) = -i a \lambda \sin \theta \pm \sqrt{1 - a^2 \lambda^2 \sin^2 \theta - \epsilon \sin^4 \frac{1}{2} \theta}.
\end{equation*}
If the expression under the $\sqrt{\cdot}$ is nonnegative, then
\begin{equation*}
\abs{g_{\pm}(\theta)}^2 = 1 - \epsilon \sin^4 \frac{1}{2} \theta \leq 1,
\end{equation*}
hence the scheme is stable. We thus wish to satisfy
\begin{equation*}
0 \leq 1 - a^2 \lambda^2 \sin^2 \theta - \epsilon \sin^4 \frac{1}{2} \theta =: \alpha(\theta).
\end{equation*}
We compute that
\begin{equation*}
\alpha'(\theta) = -\frac{1}{2} \sin \theta \left( \left( 4 a^2 \lambda^2 - \epsilon \right) \cos \theta + \epsilon \right),
\end{equation*}
and hence the extrema of $\alpha$ occur when $\sin \theta = 0$ or $\cos \theta = \epsilon / \left( \epsilon - 4 a^2 \lambda^2 \right)$. Values of $\theta$ satisfying $\sin \theta = 0$ give $\alpha = 1$ or $\alpha = 1 - \epsilon$, requiring that $\epsilon \leq 1$. Values of $\theta$ satisfying $\cos \theta = \epsilon / \left( \epsilon - 4 a^2 \lambda^2 \right)$ exist if and only if $\abs{\epsilon / \left( \epsilon - 4 a^2 \lambda^2 \right)} \leq 1$, which is equivalent to $\epsilon \leq 2 a^2 \lambda^2$. For such $\theta$, we get $\alpha = 1 - 4 a^4 \lambda^4 / \left( 4 a^2 \lambda^2 - \epsilon \right)$, and for this to be nonnegative, we must have $\epsilon \leq 4 a^2 \lambda^2 \left( 1 - a^2 \lambda^2 \right)$. In particular, we must have $\abs{a \lambda} < 1$.

So far, we have deduced that, at a minimum, $0 < \epsilon \leq 1$. Furthermore, if $\epsilon \leq 2 a^2 \lambda^2$, then we must additionally satisfy $\epsilon \leq 4 a^2 \lambda^2 \left( 1 - a^2 \lambda^2 \right)$. Now, in the instance that $2 a^2 \lambda^2 \leq 4 a^2 \lambda^2 \left( 1 - a^2 \lambda^2 \right)$, we would automatically satisfy the second condition, and this latter inequality is equivalent to $a^2 \lambda^2 \leq \frac{1}{2}$. It follows that
\begin{itemize}
\item If $0 < a^2 \lambda^2 \leq \frac{1}{2}$, it is sufficient to take $0 < \epsilon \leq 1$.
\item If $\frac{1}{2} \leq a^2 \lambda^2 < 1$, it is sufficient to take $0 < \epsilon \leq 4 a^2 \lambda^2 \left( 1 - a^2 \lambda^2 \right)$.
\end{itemize}

\item[2.] Derive the stability condition for the backward-time forward-space scheme
\begin{equation*}
\frac{1}{k} \left( v^{n+1}_m - v^n_m \right) + \frac{a}{h} \left( v^{n+1}_{m+1} - v^{n+1}_m \right) = 0
\end{equation*}
used to approximate solutions to $u_t + a u_x = 0$ with, say, $x \in [0,1]$ and periodic boundary conditions. Give an example of an initial condition $v^0_m$ and an explicit expression for $v^n_m$ that demonstrate unstable behavior for a particular $\lambda$ (your choice) which fails to satisfy the stability condition. Does the growth in your example agree with your theoretical amplification factor?

\textbf{Solution}

Our difference operator is
\begin{equation*}
P_{k,h} v^n_m = \frac{1}{k} \left( v^{n+1}_m - v^n_m \right) + \frac{a}{h} \left( v^{n+1}_{m+1} - v^{n+1}_m \right)
\end{equation*}
which has symbol
\begin{align*}
p_{k,h}(s,\xi) & = P_{k,h} \left( e^{skn + imh\xi} \right) / e^{skn + imh\xi} \\
               & = \frac{1}{k} \left( e^{sk} - 1 \right) + \frac{a}{h} e^{sk} \left( e^{ih\xi} - 1 \right).
\end{align*}
We determine stability by finding the roots of the symbol as a function of $g := e^{sk}$, yielding
\begin{equation*}
g = \frac{1}{1 + a \lambda \left( e^{i\theta} - 1 \right)}
\end{equation*}
where $\lambda := k/h$ and $\theta := h \xi$. We find that
\begin{equation*}
\abs{g}^{-2} = 1 + 2 a \lambda \left( a \lambda - 1 \right) \left( 1 - \cos \theta \right),
\end{equation*}
hence the scheme is stable ($\abs{g} \leq 1$) if and only if $a \leq 0$ or $a \lambda \geq 1$.

If, for example, $a \lambda = \frac{1}{4}$, then
\begin{equation*}
\abs{g}^{-2} = 1 - \frac{3}{8} \left( 1 - \cos \theta \right) = \frac{5}{8} + \frac{3}{8} \cos \theta.
\end{equation*}
Choosing, for example, $\theta = \pi$ ought to give an amplication factor of exactly $g = 2$ of the pure mode $v_m = e^{i \theta m} = (-1)^m$. Indeed, one can quickly verify that $v^n_m = 2^n (-1)^m$ satisfies the difference equation:
\begin{align*}
k P_{k,h} v^n_m
 & = v^{n+1}_m - v^n_m + a \lambda \left( v^{n+1}_{m+1} - v^{n+1}_m \right) \\
 & = 2^{n+1} (-1)^m - 2^n (-1)^m + \frac{1}{4} \left( 2^{n+1} (-1)^{m+1} - 2^{n+1} (-1)^m \right) \\
 & = 2^n (-1)^m \left( 2 - 1 + \frac{1}{4} \left( -2 -2 \right) \right) \\
 & = 0.
\end{align*}
One final remark: Notice that if $a \lambda = \frac{1}{2}$, $\abs{g}$ is \emph{unbounded} near $\theta = \pi$. This corresponds to a null space in the resulting system of equations for $v^{n+1}$ induced by the difference operator, and this null space is spanned precisely by the mode corresponding to $\theta = \pi$, $v_m = (-1)^m$.

\item[3.] Prove that numerical solutions to the Lax-Friedrichs scheme
\begin{equation*}
\frac{1}{k} \left( v^{n+1}_m - \frac{1}{2} \left( v^n_{m+1} + v^n_{m-1} \right) \right) + \frac{a}{2h} \left( v^n_{m+1} - v^n_{m-1} \right) = 0
\end{equation*}
converge to solutions to the corresponding modified equation
\begin{equation*}
u_t + a u_x = \frac{h^2}{2k} \left( 1 - \left( \frac{a k}{h} \right)^2 \right) u_{xx}
\end{equation*}
to second order accuracy in $L^{\infty}$. I.e., show that $\abs{v^n - u_{k,h} \left( t_n, \cdot \right)}_{\infty} \to 0$ (with $t_n = T$) as $h,k \to 0$ (according to the stability criterion), where the subscripts on $u_{k,h}$ only indicate that the solution to the modified equation is parameterized by $k,h$.

\textbf{Solution}

The difference operator for the Lax-Friedrichs scheme is
\begin{equation*}
P_{k,h} v^n_m = \frac{1}{k} \left( v^{n+1}_m - \frac{1}{2} \left( v^n_{m+1} + v^n_{m-1} \right) \right) + \frac{a}{2h} \left( v^n_{m+1} - v^n_{m-1} \right)
\end{equation*}
which has symbol
\begin{align*}
p_{k,h}(s,\xi) & = P_{k,h} \left( e^{skn + imh\xi} \right) / e^{skn + imh\xi} \\
               & = \frac{1}{k} \left( e^{sk} - \cos h \xi \right) + i \frac{a}{h} \sin h \xi \\
               & = \frac{1}{k} \left( 1 + sk + \frac{1}{2} s^2 k^2 - 1 + \frac{1}{2} h^2 \xi^2 \right) + i \frac{a}{h} \left( h \xi \right) + O \left( k^2 + h^2 + h^4 k^{-1} \right) \\
               & = s + i a \xi + \frac{k}{2} s^2 + \frac{h^2}{2k} \xi^2 + O \left( k^2 + h^2 + h^4 k^{-1} \right).
\end{align*}

\item[4.] (Strikwerda 4.1.2.) Show that the $(2,2)$ leapfrog scheme for $u_t + a u_{xxx} = f$ (see (2.2.15)) given by
\begin{equation*}
\frac{v^{n+1}_m - v^{n-1}_m}{2k} + a \delta^2 \delta_0 v^n_m = f^n_m,
\end{equation*}
with $\nu = k / h^3$ constant, is stable if and only if
\begin{equation*}
\abs{a \nu} < \frac{2}{3^{3/2}}.
\end{equation*}

\textbf{Solution}

\item[5.] (Strikwerda 3.2.1.) Show that the (forward-backward) MacCormack scheme
\begin{align*}
\tilde{v}^{n+1}_m & = v^n_m - a \lambda \left( v^n_{m+1} - v^n_m \right) + k f^n_m, \\
v^{n+1}_m & = \frac{1}{2} \left( v^n_m + \tilde{v}^{n+1}_m - a \lambda \left( \tilde{v}^{n+1}_m - \tilde{v}^{n+1}_{m-1} \right) + k f^{n+1}_m \right)
\end{align*}
is a second-order accurate scheme for the one-way wave equation (1.1.1). Show that for $f = 0$ it is identical to the Lax-Wendroff scheme (3.1.1).

\textbf{Solution}

\end{itemize}

\section{Programming}

\begin{itemize}

\item[1.] For the one-way wave equation $u_t + a u_x = 0$, investigate how close the numerical solution to a finite difference scheme is to the solution to the corresponding modified equation. To be concrete, suppose a convenient initial condition for which you can solve the modified equation explicitly with periodic boundary conditions. Take $a = 1$, $k/h = 0.5$, and final time $T = 0.5$. Compare the following finite difference schemes: upwinding, Lax-Friedrichs, and Lax-Wendroff. Also, include a derivation of the respective corresponding modified equations.

\end{itemize}

\end{document}
