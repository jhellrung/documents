\documentclass{article}

%\usepackage[left=1in,top=1in,bottom=1in,right=1in,nohead,nofoot]{geometry}
\usepackage{fullpage}
\usepackage{amsmath}
\usepackage{amsfonts}
\usepackage{graphicx}



\def\supp{\mathop{\rm supp}\nolimits}


\begin{document}


\begin{flushright}
Jeffrey Hellrung \\
Applied Differential Equations Qualifying Exam, Fall 2006 \\
\end{flushright}


\begin{enumerate}

\item Consider the second-order ODE
\[x''(t) + x^3(t) - 4 x(t) = 0. \ \ \ \ (1)\]

\begin{itemize}
\item Find the conserved quantity for (1).

\item Rewrite (1) as a first-order system.

\item Find and classify the equilibrium points.

\item Sketch the phase portrait of the systems.

\end{itemize}

{\bf Solution}

\begin{itemize}
\item Multiplying by \(x'\) and integrating gives
\[C = \left( x' \right)^2 + \frac{1}{4} x^4 - 2 x^2.\]

\item
\[(x,x')' = (x', 4 x - x^3) = F(x,x').\]

\item Equilibrium points \((x,x')^*\) satisfy
\[F((x,x')^*) = 0 \ \Rightarrow \ \ (x,x')^* \in \{(0,0), (\pm 2, 0)\}.\]
To classify the equilibrium points, we compute
\[DF(x,x') = \left( \begin{array}{cc} 0 & 1 \\ 4 - 3 x^2 & 0 \end{array} \right).\]

\begin{itemize}
\item \((x,x')^* = (0,0)\):
\[DF(0,0) = \left( \begin{array}{cc} 0 & 1 \\ 4 & 0 \end{array} \right)\]
has eigenvalues \(\lambda_{\pm} = \pm 2\) and corresponding eigenvectors
\[v_{\pm} = \left( \begin{array}{c} \pm 1 \\ 2 \end{array} \right).\]
This equilibrium point is a saddle.

\item \((x,x')^* = (2,0)\):
\[DF(2,0) = \left( \begin{array}{cc} 0 & 1 \\ -8 & 0 \end{array} \right)\]
has eigenvalues \(\lambda_{\pm} = \pm 2 \sqrt{2} i\).  This equilibrium point is a center.

\item \((x,x')^* = (-2,0)\):  [Same as previous case.]

\end{itemize}

\item

\end{itemize}



\item Consider the equation
\[u_{tt} = c^2 u_{xx} \ \ \ \ (2)\]
for \(-a t < x < a t\) and \(0 \leq t\), in which \(a\) and \(c\) are positive constants.  For which boundary conditions on \(x = \pm a t\) is there existence and uniqueness for this problem?  Hint:  The answer depends on \(a\).

{\bf Solution}





\item Consider the PDE
\begin{eqnarray*}
u_t & = & \Delta u; \ \ \ \ (3) \\
u(x,y,t = 0) & = & u_0(x,y); \ \ \ \ (4)
\end{eqnarray*}
in a half-plane \(-\infty < x < \infty\) and \(0 \leq y < \infty\), with \(u_0(x,y) \geq 0\).  Compare the following two boundary conditions:
\[u(x,0,t) = 0 \ \ \ \ (5)\]
and
\[u_y(x,0,t) = 0. \ \ \ \ (6)\]
Denote the solution of (3),(4), and (5) as \(u^D\); and the solution of (3), (4), and (6) as \(u^N\).  Show that \(u^D \leq u^N\) for all \(x,y\) and \(t > 0\).

{\bf Solution}

To obtain \(u^D\), we can extend \(u_0(x,y)\) for \(y < 0\) ``oddly'' by setting \(u_0(x,y) = -u_0(x,-y)\), yielding
\begin{eqnarray*}
u^D(x,y) & = & \frac{1}{4 \pi t} \iint_{\mathbb{R}^2} e^{-\left( (x - \xi)^2 + (y - \eta)^2 \right) / 4 t} u_0(\xi,\eta) d\xi d\eta \\
         & = & \frac{1}{4 \pi t} \iint_{\eta \geq 0} \left( e^{-\left( (x - \xi)^2 + (y - \eta)^2 \right) / 4 t} u_0(\xi,\eta) - e^{-\left( (x - \xi)^2 + (y + \eta)^2 \right) / 4 t} u_0(\xi,\eta) \right) d\xi d\eta \\
         & = & \frac{1}{4 \pi t} \iint_{\eta \geq 0} e^{-\left( (x - \xi)^2 + (y - \eta)^2 \right) / 4 t} \left( 1 - e^{-y \eta / t} \right) u_0(\xi,\eta) d\xi d\eta.
\end{eqnarray*}
Similarly, to obtain \(u^N\), we can extend \(u_0(x,y)\) for \(y < 0\) ``evenly'' by setting \(u_0(x,y) = u_0(x,-y)\), yielding
\begin{eqnarray*}
u^N(x,y) & = & \frac{1}{4 \pi t} \iint_{\mathbb{R}^2} e^{-\left( (x - \xi)^2 + (y - \eta)^2 \right) / 4 t} u_0(\xi,\eta) d\xi d\eta \\
         & = & \frac{1}{4 \pi t} \iint_{\eta \geq 0} e^{-\left( (x - \xi)^2 + (y - \eta)^2 \right) / 4 t} \left( 1 + e^{-y \eta / t} \right) u_0(\xi,\eta) d\xi d\eta.
\end{eqnarray*}
Since \(u_0 \geq 0\), it is easy to see that \(u^N \geq u^D\).



\item Consider the Laplace equation
\[\frac{\partial^2 u}{\partial x^2} + \frac{\partial^2 u}{\partial y^2} = 0, \ y > 0, \ x \in \mathbb{R}, \ \ \ \ (7)\]
together with the boundary condition
\[\frac{\partial u}{\partial y}(x,0) - u(x,0) = f(x), \ \ \ \ (8)\]
where \(f(x) \in C_0^{\infty}(\mathbb{R})\) (i.e., \(f\) is smooth with compact support).  Find a representation for a bounded solution \(u(x,y)\) of (7), (8); and show that \(u(x,y) \to 0\) as \(y \to \infty\) uniformly in \(x \in \mathbb{R}\).

{\bf Solution}

We apply a Fourier transform in \(x\).  Recall that the Fourier transform, at least formally, is given by
\[\mathcal{F}_x(f(x))(\xi) = \int_{-\infty}^{\infty} e^{-i x \xi} f(x) dx,\]
and it is easy to verify that
\[\mathcal{F}_x(f'(x))(\xi) = i \xi \mathcal{F}_x(f(x))(\xi).\]
Denoting by \(\widehat{u}(\xi,y) = \mathcal{F}_x(u(x,y))(\xi)\), we see that \(\widehat{u}\) satisfies
\[-\xi^2 \widehat{u} + \widehat{u}_{yy} = 0\]
subject to the boundary condition
\[\widehat{u}_y(\xi,0) - \widehat{u}(\xi,0) = \mathcal{F}_x(f(x))(\xi) = \widehat{f}(\xi).\]
We can find the general solution for \(\widehat{u}\):
\[\widehat{u}(\xi,y) = C_1(\xi) e^{-|\xi| y} + C_2(\xi) e^{|\xi| y}.\]
Boundedness requires \(C_2 = 0\), while the boundary conditions require
\[C_1(\xi) = -\frac{\widehat{f}(\xi)}{1 + |\xi|},\]
and therefore
\[\widehat{u}(\xi,y) = -\frac{\widehat{f}(\xi)}{1 + |\xi|} e^{-|\xi| y}.\]
Since \(f \in C_0^{\infty} \subset \mathcal{S}\), \(\widehat{f} \in \mathcal{S}\), from which it follows easily that \(\xi \mapsto \widehat{u}(\xi,y) \in \mathcal{S}\), so \(x \mapsto u(x,y) \in \mathcal{S}\).  The exponential decrease in \(y\) also implies that \(y \mapsto u(x,y) \in \mathcal{S}\) as well, hence \(u\) is bounded.  Indeed,
\begin{eqnarray*}
|u(x,y)| &   =  & \left| \frac{1}{2 \pi} \int_{-\infty}^{\infty} e^{i x \xi} \widehat{u}(\xi,y) d\xi \right| \\
         &   =  & \frac{1}{2 \pi} \left| \int_{-\infty}^{\infty} e^{i x \xi} \frac{\widehat{f}(\xi)}{1 + |\xi|} e^{-|\xi| y} d\xi \right| \\
         & \leq & \frac{1}{2 \pi} \left\| \widehat{f} \right\|_{L^2} \left\| \frac{e^{-|\xi| y}}{1 + |\xi|} \right\|_{L^2_{\xi}}
\end{eqnarray*}
by the Cauchy-Schwarz Inequality.  But
\[\left\| \widehat{f} \right\|_{L^2} = \frac{1}{2 \pi} \|f\|_{L^2} < \infty,\]
by Plancherel's Theorem, while
\begin{eqnarray*}
\left\| \frac{e^{-|\xi| y}}{1 + |\xi|} \right\|_{L^2_{\xi}}^2
&   =  & 2 \int_0^{\infty} \left( \frac{e^{-\xi y}}{1 + \xi} \right)^2 d\xi \\
& \leq & 2 \int_0^{\infty} e^{-2 \xi y} d\xi \\
&   =  & \frac{1}{y},
\end{eqnarray*}
from which it follows that
\[|u(x,y)| \leq \frac{1}{4 \pi^2} \|f\|_{L^2} y^{-1} \to 0\]
as \(y \to \infty\), uniformly in \(x\).



\item Let \(a \in \mathbb{R}\) be a positive constant and \(f(t)\) a non-negative continuous function.  Assume that \(y(t)\) is a continuous function such that
\[0 \leq y(t) \leq a + \int_0^t f(s) y(s)^2 ds \ \ \text{for \(t \geq 0\)}.\]
Show that
\[y(t) \leq \frac{a}{1 - a \int_0^t f(s) ds} \ \ \ \ (10)\]
for all \(t \geq 0\) for which the denominator in the right hand side of (10) is positive.

{\bf Solution}

Denote by
\[z(t) = a + \int_0^t f(s) y(s)^2 ds,\]
and note that
\[z'(t) = f(t) y(t)^2 \leq f(t) z(t)^2.\]
It follows that
\[\int_0^t \frac{z'(s)}{z(s)^2} ds \leq \int_0^t f(s) ds \ 
  \Rightarrow \ \frac{1}{z(0)} - \frac{1}{z(t)} \leq \int_0^t f(s) ds \ 
  \Rightarrow \ z(t) \leq \frac{a}{1 - a \int_0^t f(s) ds},\]
from which the claim follows.



\item Let \(\phi \in C^1(\mathbb{C})\) be a function with compact support.  When \(\zeta \in \mathbb{C}\), let us write \(\zeta = \xi + i \eta\), with \(\xi,\eta \in \mathbb{R}\), and introduce the Cauchy-Riemann operator,
\[\frac{\partial}{\partial \overline{\zeta}} = \frac{1}{2} \left( \frac{\partial}{\partial \xi} + i \frac{\partial}{\partial \eta} \right).\]
Let \(z \in \mathbb{C}\).  Show that
\[\phi(z) = -\frac{1}{\pi} \iint \frac{\partial \phi}{\partial \overline{\zeta}} (\zeta) (\zeta - z)^{-1} d\xi d\eta.\]

{\bf Solution}

Notice that
\[\Delta = 4 \frac{\partial}{\partial \zeta} \frac{\partial}{\partial \overline{\zeta}},\]
where
\[\frac{\partial}{\partial \zeta} = \frac{1}{2} \left( \frac{\partial}{\partial \xi} - i \frac{\partial}{\partial \eta} \right).\]
Further, if \(K(\xi,\eta) = \frac{1}{4 \pi} \log (\xi^2 + \eta^2)\) is the fundamental solution to the Laplacian, that is,
\[\Delta K = \delta,\]
then we can easily compute that
\[\frac{\partial}{\partial \zeta} K(\zeta) = \frac{1}{4 \pi} \zeta^{-1},\]
so that, using integration by parts,
\begin{eqnarray*}
\phi(z) & = & \iint \phi(\zeta) \Delta K(z - \zeta) d\xi d\eta \\
        & = & -4 \iint \frac{\partial \phi}{\partial \overline{\zeta}}(\zeta) \frac{\partial}{\partial \zeta} K(z - \zeta) d\xi d\eta \\
        & = & -\frac{1}{\pi} \iint \frac{\partial \phi}{\partial \overline{\zeta}}(\zeta) (\zeta - z)^{-1} d\xi d\eta.
\end{eqnarray*}




\item Let \(u\) solve the heat equation in a two-dimensional channel, i.e.,
\begin{eqnarray*}
u_t & = & \Delta u; \\
u(x,y,t = 0) & = & u_0(x,y); \\
u_y(x,0,t) = u_y(x,\pi,t) & = & 0;
\end{eqnarray*}
for \(-\infty < x < \infty\) and \(0 \leq y \leq \pi\).  The initial data \(u_0\) is assumed to be smooth and vanish for \(|x|\) large.

\begin{enumerate}
\item Show that \(u(x,y,t)\) can be expanded in a cosine series in \(y\), i.e.,
\[u(x,y,t) = \sum_{k \geq 0} \widehat{u}(x,k,t) \cos(k y)\]
and find an equation for the \(k^{th}\) coefficient \(\widehat{u}(x,k,t)\).

\item Find the limit of \(t^{1/2} u(x,y,t)\) as \(t \to \infty\).

\end{enumerate}

{\bf Solution}

\begin{enumerate}
\item We separate variables by assuming \(u(x,y,t) = Y(y) Z(x,t)\) and substituting into the PDE, finding that
\[0 = u_t - \Delta u = Y Z_t - Y Z_{xx} - Y'' Z \ \Rightarrow \ \frac{Z_t - Z_{xx}}{Z} = \frac{Y''}{Y} = \lambda\]
for some constant \(\lambda\).  Solving for \(Y\) yields the general solution \(Y = C_1 e^{\sqrt{\lambda} y} + C_2 e^{\sqrt{\lambda} y}\).  The boundary conditions \(u_y(x,0,t) = u_y(x,\pi,t) = 0\) imply that \(Y'(0) = Y'(\pi) = 0\), hence we conclude that \(Y = \cos \left( \sqrt{-\lambda} y \right)\), with \(\lambda \leq 0\) and \(\sqrt{-\lambda} \in \mathbb{Z}\).  Letting \(\lambda = -k^2\) for \(k \in \mathbb{Z}\), this gives simply \(Y_k = \cos(k y)\).  By linearity, then, \(u\) must be of the form
\[u(x,t) = \sum_{k \geq 0} Z_k(x,t) \cos(k y),\]
where \(Z_k\) satisfies
\[(Z_k)_t - (Z_k)_{xx} = -k^2 Z_k, \ x \in \mathbb{R}, \ t \geq 0.\]
This has the solution
\[Z_k(x,t) = \frac{1}{\sqrt{4 \pi t}} e^{-k^2 t} \int_{-\infty}^{\infty} e^{-(x - \xi)^2 / 4 t} Z_k(\xi,0) d\xi.\]
We notice that \(Z_k(x,0)\) is just the coefficient on \(\cos(k y)\) when expanding \(u_0(x,y)\) for fixed \(x\):
\[Z_k(x,0) = \frac{2}{\pi} \int_0^{\pi} u_0(x,y) \cos(k y) dy.\]

\item In the limit as \(t \to \infty\), \(Z_0\) dominates \(Z_k\) for \(k > 0\), due to the presence of the exponentially decaying factor \(e^{-k^2 t}\).  Thus,
\[\lim_{t \to \infty} \sqrt{t} u(x,y,t) = \lim_{t \to \infty} \sqrt{t} Z_0(x,t).\]
Noting that
\[Z_0(x,0) = \frac{2}{\pi} \int_0^{\pi} u_0(x,y) dy,\]
we thus obtain
\begin{eqnarray*}
\sqrt{t} Z_0(x,t)
&  =  & \frac{1}{\pi^{3/2}} \int_{-\infty}^{\infty} e^{-(x - \xi)^2 / 4 t} \left( \int_0^{\pi} u_0(\xi,y) dy \right) d\xi \\
& \to & \frac{1}{\pi^{3/2}} \int_{-\infty}^{\infty} \int_0^{\pi} u_0(\xi,y) dy d\xi
\end{eqnarray*}
as \(t \to \infty\).

\end{enumerate}



\item Suppose that \(u\) is a smooth solution of the initial boundary value problem
\begin{eqnarray*}
u_t & = & u_{xx} + c u^2; \\
u(x,t = 0) & = & u_0(x); \\
u(0,t) & = & u(1,t) = 0; \ \ \ \ (18)
\end{eqnarray*}
for \(0 < x < 1\), in which \(c\) is a positive constant.

\begin{enumerate}
\item Show that
\[\frac{d}{dt} \int_0^1 |u(x,t)|^2 dx \leq -\left( \int_0^1 |u_x(x,t)|^2 dx \right) \left( 1 - c \left( \int_0^1 |u(x,t)|^2 dx \right)^{1/2} \right).\]
Hint:  First show that
\[\sup_x |u(x,t)|^2 \leq \int_0^1 |u_x(x,t)|^2 dx.\]

\item If the initial data \(u_0\) satisfies
\[\int_0^1 |u_0(x)|^2 dx < \frac{1}{c^2},\]
show that \(u\) satisfies
\[\int_0^1 |u(x,t)|^2 dx < \frac{1}{c^2}\]
for all time.  Hint:  Show that
\[\frac{d}{dt} \int_0^1 |u(x,t)|^2 dx \leq 0.\]

\item If the boundary condition (18) is changed to \(\partial_x u_0 = 0\) at \(x = 0\) and \(x = 1\), find a counterexample, i.e., find initial data \(u_0\) for which the solution blows up in finite time.

\end{enumerate}

{\bf Solution}

\begin{enumerate}
\item We first show the suggested hint:
\begin{eqnarray*}
u(x,t) & \leq & |u(x,t)| \\
       &   =  & \left| \int_0^x u_x(y,t) dy \right| \\
       & \leq & \int_0^x |u_x(y,t)| dy \\
       & \leq & \int_0^1 |u_x(x,t)| dx \\
       & \leq & \left( \int_0^1 |u_x(x,t)|^2 dx \right)^{1/2},
\end{eqnarray*}
where the last inequality is from an application of the Cauchy-Schwarz Inequality.  It follows that
\begin{eqnarray*}
\frac{d}{dt} \int_0^1 |u(x,t)|^2 dx
&   =  & \int_0^1 \frac{d}{dt} |u(x,t)|^2 dx \\
&   =  & \int_0^1 2 \Re \left( \overline{u(x,t)} u_t(x,t) \right) dx \\
&   =  & 2 \Re \left( \int_0^1 \overline{u(x,t)} u_t(x,t) dx \right) \\
&   =  & 2 \Re \left( \int_0^1 \overline{u(x,t)} \left( u_{xx}(x,t) + c u(x,t)^2 \right) dx \right) \\
&   =  & 2 \Re \left( \int_0^1 \overline{u(x,t)} u_{xx}(x,t) dx + c \int_0^1 |u(x,t)|^2 u(x,t) dx \right) \\
&   =  & 2 \Re \left( -\int_0^1 |u_x(x,t)|^2 dx + c \int_0^1 |u(x,t)|^2 u(x,t) dx \right) \\
& \leq & 2 c \int_0^1 |u(x,t)|^3 dx - 2 \int_0^1 |u_x(x,t)|^2 dx \\
& \leq & 2 \left( \int_0^1 |u_x(x,t)|^2 dx \right) \left( c \int_0^1 |u(x,t)| dx - 1 \right) \\
& \leq & 2 \left( \int_0^1 |u_x(x,t)|^2 dx \right) \left( c \left( \int_0^1 |u(x,t)|^2 \right)^{1/2} - 1 \right).
\end{eqnarray*}

\item Let
\[T = \inf \left\{ t \geq 0 \ \left| \ \int_0^1 |u(x,t)|^2 dx \geq \frac{1}{c^2} \right. \right\}.\]
By continuity of \(u\), \(T > 0\), while for \(t \leq T\), the inequality from (a) gives that
\[\frac{d}{dt} \int_0^1 |u(x,t)|^2 dx \leq 0.\]
This implies that
\[\int_0^1 |u(x,t)|^2 dx \leq \int_0^1 |u(x,0)|^2 dx < \frac{1}{c^2}\]
for all \(t \leq T\), particularly for \(t = T\) if \(T < \infty\).  It follows that we must have \(T = \infty\), hence
\[\int_0^1 |u(x,t)|^2 dx < \frac{1}{c^2}\]
for all \(t \geq 0\).

\item If we take \(u_0(x) = \alpha\), then it is easy to verify that the solution to the new boundary value problem is
\[u(x,t) = \frac{\alpha}{1 - c \alpha t},\]
which blows up at \(t = (c \alpha)^{-1}\).

\end{enumerate}




\end{enumerate}

\end{document}
