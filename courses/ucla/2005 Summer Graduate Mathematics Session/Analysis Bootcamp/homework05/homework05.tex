\documentclass{article}

%\usepackage[left=1in,top=1in,bottom=1in,right=1in,nohead,nofoot]{geometry}
\usepackage{fullpage}
\usepackage{amsmath}
\usepackage{amsfonts}
\usepackage{graphicx}


\begin{document}


\begin{flushright}
Jeffrey Hellrung \\
Monday, September 12, 2005 \\
Analysis Bootcamp, Homework 5 \\
\end{flushright}


\begin{enumerate}

\item {\em Rudin, page 139.  Problems 10 - 12, 15, 17, 19.}

\begin{itemize}

\item[10.] {\em Let \(p\) and \(q\) be positive real numbers such that
\[\frac{1}{p} + \frac{1}{q} = 1.\]
Prove the following statements.
\begin{enumerate}
\item If \(u \geq 0\) and \(v \geq 0\), then
\[uv \leq \frac{u^p}{p} + \frac{v^q}{q}.\]
Equality holds if and only if \(u^p = v^q\).
\item If \(f \in \mathcal{R}(\alpha)\), \(g \in \mathcal{R}(\alpha)\), \(f \geq 0\), \(g \geq 0\), and
\[\int_a^b f^p d\alpha = 1 = \int_a^b g^q d\alpha,\]
then
\[\int_a^b fg d\alpha \leq 1.\]
\item If \(f\) and \(g\) are complex functions in \(\mathcal{R}(\alpha)\), then
\[\left| \int_a^b fg d\alpha \right| \leq \left( \int_a^b |f|^p d\alpha \right)^{1/p} \left( \int_a^b |g|^q d\alpha \right)^{1/q}.\]
This is {\em H\"older's inequality}.  When \(p = q = 2\) it is usually called the Schwarz inequality.  (Note that Theorem 1.35 is a very special case of this.)
\item Show that H\"older's inequality is also true for the ``improper'' integrals described in Exercises 7 and 8.
\end{enumerate}}

{\bf Solution}

\begin{enumerate}
\item The inequality is trivial when either \(u = 0\) or \(v = 0\), so assume both are positive.  \(z \to e^z\) is a concave up, hence for any \(x,y \in \mathbb{R}\) and \(t \in [0,1]\),
\[(1 - t) e^x + t e^y \geq e^{(1 - t)x + ty} = \left( e^x \right)^{1 - t} \left( e^y \right)^t.\]
Setting \(x = \ln u^p\), \(y = \ln v^q\), and \(t = \frac{1}{q}\) establishes the inequality.  Note that equality is achieved when either \(t \in \{0,1\}\) (which, for positive \(p,q\), cannot happen) or when \(x = y\), which happens if and only if \(u^p = v^q\).

\item Using the above inequality, for all \(x \in [a,b]\),
\[\frac{f(x)^p}{p} + \frac{g(x)^q}{q} \geq f(x) g(x),\]
hence, given \(\int f^p d\alpha = \int g^q d\alpha = 1\),
\[1 = \int_a^b \left( \frac{f^p}{p} + \frac{g^q}{q} \right) d\alpha \geq \int_a^b fg d\alpha.\]

\item Let
\[c = \left( \int_a^b |f|^p d\alpha \right)^{1/p},\]
\[d = \left( \int_a^b |g|^q d\alpha \right)^{1/q}.\]
Then \(\int \left| \frac{f}{c} \right|^p d\alpha = \int \left| \frac{g}{d} \right|^q d\alpha = 1\), hence the preceding inequality gives
\[\int_a^b \left| \frac{f}{c} \right| \left| \frac{g}{d} \right| d\alpha \leq 1\]
from which it follows that
\[\int_a^b |f| |g| d\alpha \leq cd = \left( \int_a^b |f|^p d\alpha \right)^{1/p} \left( \int_a^b |g|^q d\alpha \right)^{1/q}.\]

\item

\end{enumerate}



\item[11.] {\em Let \(\alpha\) be a fixed increasing function on \([a,b]\).  For \(u \in \mathcal{R}(\alpha)\), define
\[\|u\|_2 = \left( \int_a^b |u|^2 d\alpha \right)^{1/2}.\]
Suppose \(f,g,h \in \mathcal{R}(\alpha)\), and prove that triangle inequality
\[\|f - h\|_2 \leq \|f - g\|_2 + \|g - h\|_2\]
as a consequence of the Schwarz inequality, as in the proof of Theorem 1.37.}

{\bf Solution}

\[\|f - h\|_2^2
     = \int |f - h|^2 d\alpha
     = \int \left( |f|^2 + |h|^2 - f\overline{h} - \overline{f}h \right) d\alpha\]
\[   = \int \left( |f|^2 + |g|^2 - f\overline{g} - \overline{f}g \right) d\alpha
     + \int \left( |g|^2 + |h|^2 - g\overline{h} - \overline{g}h \right) d\alpha\]
\[   + \int \left( f\overline{g} + g\overline{h} - f\overline{h} - |g|^2 \right) d\alpha
     + \int \left( \overline{f}g + \overline{g}h - \overline{f}h - |g|^2 \right) d\alpha\]
\[   = \int |f - g|^2 d\alpha + \int |g - h|^2 d\alpha
     + \int (f - g) \overline{(g - h)} d\alpha + \int \overline{(f - g)} (g - h) d\alpha\]
\[\leq \|f - g\|_2^2 + \|g - h\|_2^2
     + 2 \left( \int |f - g|^2 d\alpha \right)^{1/2} \left( \int |g - h|^2 d\alpha \right)^{1/2}\]
\[   = \|f - g\|_2^2 + \|g - h\|_2^2 + 2 \|f - g\|_2 \|g - h\|_2\]
\[   = \left( \|f - g\|_2 + \|g - h\|_2 \right)^2.\]



\item[12.] {\em With the notations of Exercise 11, suppose \(f \in \mathcal{R}(\alpha)\) and \(\epsilon > 0\).  Prove that there exists a continuous function \(g\) on \([a,b]\) such that \(\|f - g\|_2 < \epsilon\).

{\em Hint:}  Let \(P = \{x_0, \ldots, x_n\}\) be a suitable partition of \([a,b]\), define
\[g(t) = \frac{x_i - t}{\Delta x_i} f(x_{i - 1}) + \frac{t - x_{i - 1}}{\Delta x_i} f(x_i)\]
if \(x_{i - 1} \leq t \leq x_i\).}

{\bf Solution}

Since \(f \in \mathcal{R}(\alpha)\), \(f\) is bounded, say \(|f| \leq M\).  Given \(\epsilon > 0\), choose \(P = \{x_0, \ldots, x_n\}\) a partition of \([a,b]\) according to Theorem 6.6 for \(\epsilon / 2M\):
\[U(P,f,\alpha) - L(P,f,\alpha) < \epsilon / 2M.\]
Set \(g\) as in the hint, i.e.,
\[g(t) = \frac{x_i - t}{\Delta x_i} f(x_{i - 1}) + \frac{t - x_{i - 1}}{\Delta x_i} f(x_i)\]
for \(t \in [x_{i - 1}, x_i]\).  Then \(g\) is continuous (since \(g(x_i) = f(x_i)\) whether we consider \(t = x_i\) to be in \([x_{i - 1}, x_i]\) or \([x_i, x_{i + 1}]\)), and for \(t \in [x_{i - 1}, x_i]\), \(g(t)\) is between \(f(x_{i - 1})\) and \(f(x_i)\), hence \(m_i \leq g(t) \leq M_i\).  It follows that
\[m_i \Delta\alpha_i \leq g(t) \Delta\alpha_i \leq M_i \Delta\alpha_i, \ t \in [x_{i - 1}, x_i],\]
hence
\[\int_a^b |f(t) - g(t)| d\alpha \leq \sum_i (M_i - m_i) \Delta\alpha_i < \epsilon / 2M,\]
and since \(|f(t) - g(t)| \leq 2M\) (\(g(t)\) is bounded in absolute value by \(M\) as well since it is bounded by images of \(f\) over each interval in \(P\)), we have that
\[\|f - g\|_2^2 = \int_a^b |f(t) - g(t)|^2 d\alpha < \epsilon.\]




\item[15.] {\em Suppose \(f\) is a real, continuously differentiable function on \([a,b]\), \(f(a) = f(b) = 0\), and
\[\int_a^b f^2(x) dx = 1.\]
Prove that
\[\int_a^b x f(x) f'(x) dx = -\frac{1}{2}\]
and that
\[\int_a^b \left( f'(x) \right)^2 dx \cdot \int_a^b x^2 f^2(x) dx > \frac{1}{4}.\]}

{\bf Solution}

By Theorem 6.22 (Integration by Parts) for \(F(x) = x f(x)\) and \(g(x) = f'(x)\),
\[\int_a^b x f(x) f'(x) dx
  = \left. x f(x) f(x) \right|_a^b - \int_a^b (f(x) + x f'(x)) f(x) dx
  = -1 - \int_a^b x f(x) f'(x),\]
hence
\[\int_a^b x f(x) f'(x) dx = -\frac{1}{2}.\]
By Cauchy-Schwarz,
\[\int_a^b \left( f'(x) \right)^2 dx \cdot \int_a^b \left( x f(x) \right)^2 dx
  \geq \left( \int_a^b f'(x) x f(x) dx \right)^2 = \frac{1}{4}\]
with equality if and only if
\[f'(x) = c x f(x)\]
for some \(c \in \mathbb{R}\).  This is a separable differential equation with solution
\[f(x) = C e^{cx^2/2}.\]
Now since \(x \mapsto e^{cx^2/2}\) is never zero, the condition that \(f(a) = f(b) = 0\) forces \(C = 0\), hence \(f \equiv 0\) and \(\int f^2 dx = 0\) for any \([a,b]\), violating the givens.  We conclude that equality is impossible, and the inequality is strict.



\item[17.] {\em Suppose \(\alpha\) increases monotonically on \([a,b]\), \(g\) is continuous, and \(g(x) = G'(x)\) for \(a \leq x \leq b\).  Prove that
\[\int_a^b \alpha(x) g(x) dx = G(b) \alpha(b) - G(a) \alpha(a) - \int_a^b G d\alpha.\]
{\em Hint:}  Take \(g\) real, without loss of generality.  Given \(P = \{x_0, x_1, \ldots, x_n\}\), choose \(t_i \in (x_{i - 1}, x_i)\) so that \(g(t_i) \Delta x_i = G(x_i) - G(x_{i - 1})\).  Show that
\[\sum_{i = 1}^n \alpha(x_i) g(t_i) \Delta x_i = G(b) \alpha(b) - G(a) \alpha(a) - \sum_{i = 1}^n G(x_{i - 1}) \Delta \alpha_i.\]}

{\bf Solution}

Let \(\epsilon > 0\) be given.  Let \(M = \max \{|\alpha(a)|, |\alpha(b)|\}\) (thus \(\alpha \leq M\) on \([a,b]\)).  Since \(g\) is continuous on \([a,b]\), which is compact, \(g\) is uniformly continuous on \([a,b]\).  Similarly, \(G\) is differentiable on \([a,b]\), hence continuous, hence uniformly continuous.  Hence there exists a \(\delta > 0\) such that \(|g(x) - g(y)| < \epsilon / M(b - a)\) and \(|G(x) - G(y)| < \epsilon / (\alpha(b) - \alpha(a))\) whenever \(|x - y| < \delta\).

Let \(P = \{x_0, \ldots, x_n\}\) be a partition as in Theorem 6.6, such that
\[U(P, \alpha g) - L(P, \alpha g) < \epsilon,\]
\[U(P, G, \alpha) - L(P, G, \alpha) < \epsilon,\]
and also such that \(\Delta x_i < \delta\) for each \(i\).  As per the hint, we can choose \(t_i \in [x_{i - 1}, x_i]\) such that \(g(t_i) \Delta x_i = G(x_i) - G(x_{i - 1})\).  Then
\[\sum_{i = 1}^n \alpha(x_i) g(t_i) \Delta x_i
   = \sum_{i = 1}^n \alpha(x_i) \left( G(x_i) - G(x_{i - 1}) \right)\]
\[ = \sum_{i = 1}^n G(x_i) \alpha(x_i) - \sum_{i = 1}^n G(x_{i - 1}) \alpha(x_i)\]
\[ = \sum_{i = 2}^{n + 1} G(x_{i - 1}) \alpha(x_{i - 1}) - \sum_{i = 1}^n G(x_{i - 1}) \alpha(x_i)\]
\[ = G(x_n) \alpha(x_n) - \sum_{i = 1}^n G(x_{i - 1}) \left( \alpha(x_i) - \alpha(x_{i - 1}) \right) - G(x_0) \alpha(x_0)\]
\[ = G(b) \alpha(b) - G(a) \alpha(a) - \sum_{i = 1}^n G(x_{i - 1}) \Delta\alpha_i.\]
Further,
\[\left| \sum_i \alpha(x_i) g(t_i) \Delta x_i - \sum_i \alpha(x_i) g(x_i) \Delta x_i \right|
 \leq \sum_i \left| \alpha(x_i) \right| \left| g(t_i) - g(x_i) \right| \Delta x_i\]
\[  < \frac{\epsilon}{M(b - a)} \sum_i \left| \alpha(x_i) \right| \Delta x_i
 \leq \frac{\epsilon}{b - a} \sum_i \Delta x_i
    = \epsilon,\]
while
\[\left| \sum_i G(x_{i - 1}) \Delta\alpha_i - \sum_i G(x_i) \Delta\alpha_i \right|
 \leq \sum_i \left| G(x_{i - 1}) - G(x_i) \right| \Delta\alpha_i
    < \frac{\epsilon}{\alpha(b) - \alpha(a)} \sum_i \Delta\alpha_i
    = \epsilon,\]
Hence we find that
\[\left| \int_a^b \alpha(x) g(x) dx - G(b) \alpha(b) + G(a) \alpha(a) + \int_a^b G d\alpha \right| < 4 \epsilon,\]
and since \(\epsilon\) was arbitrary, the equality is proved.



\item[19.] {\em Let \(\gamma_1\) be a curve in \(\mathbb{R}^k\), defined on \([a,b]\); let \(\phi\) be a continuous 1-1 mapping of \([c,d]\) onto \([a,b]\), such that \(\phi(c) = a\); and define \(\gamma_2(s) = \gamma_1(\phi(s))\).  Prove that \(\gamma_2\) is an arc, a closed curve, or a rectifiable curve if and only if the same is true of \(\gamma_1\).  Prove that \(\gamma_2\) and \(\gamma_1\) have the same length.}

{\bf Solution}

\(\gamma_1\) is an arc if and only if \(\gamma_1\) is one-to-one if and only if \(\gamma_2 = \gamma_1 \circ \phi\) is one-to-one if and only if \(\gamma_2\) is an arc.

\(\gamma_1\) is a closed curve if and only if \(\gamma_1(a) = \gamma_1(b)\) if and only if \(\gamma_2(c) = \gamma_1(a) = \gamma_1(b) = \gamma_2(d)\) if and only if \(\gamma_2\) is a closed curve.

Given \(Q = \{c = x_0, x_1, \ldots, x_{n - 1}, x_n = d\}\) a partition of \([c,d]\), there exists a corresponding partition \(P = \phi(Q)\) of \([a,b]\).  Further, if we let \(y_i = \phi(x_i)\),
\[\Lambda(P,\gamma_1) = \sum_{i = 1}^n \left| \gamma_1(y_i) - \gamma_1(y_{i - 1}) \right|
                      = \sum_{i = 1}^n \left| \gamma_2(x_i) - \gamma_2(x_{i - 1}) \right| = \Lambda(Q,\gamma_2).\]
Conversely, given a partition \(P\) of \([a,b]\), there exists a partition \(Q = \phi^{-1}(P)\) of \([c,d]\) such that \(\Lambda(Q,\gamma_2) = \Lambda(P,\gamma_1)\) (\(\phi^{-1}\) exists since \(\phi\) is one-to-one and onto).  It follows that
\[\Lambda(\gamma_1) = \sup_P \Lambda(P,\gamma_1) = \sup_Q \Lambda(Q,\gamma_1) = \Lambda(\gamma_2).\]



\end{itemize}



\end{enumerate}

\end{document}
