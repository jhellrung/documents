\documentclass{article}

%\usepackage[left=1in,top=1in,bottom=1in,right=1in,nohead,nofoot]{geometry}
\usepackage{fullpage}
\usepackage{amsmath}
\usepackage{amsfonts}
\usepackage{graphicx}


\def\im{\mathop{\rm im}\nolimits}
\def\ker{\mathop{\rm ker}\nolimits}
\def\rank{\mathop{\rm rank}\nolimits}
\def\span{\mathop{\rm span}\nolimits}
\newcommand{\matrixiiibyiii}[9]{\left( \begin{array}{ccc} #1 & #2 & #3 \\ #4 & #5 & #6 \\ #7 & #8 & #9 \end{array} \right)}

\begin{document}


\begin{flushright}
Jeffrey Hellrung \\
Basic Examination, F02 \\
\end{flushright}


\begin{enumerate}

\item Let \(K\) be a compact subset and \(F\) be a closed subset in the metric space \(X\).  Suppose \(K \cap F = \emptyset\).  Prove that
\[0 < \inf \{ d(x,y) : x \in K, y \in F \}.\]

{\bf Solution}

Let \(f : K \to \mathbb{R}\) be defined by
\[f(x) = \inf_{y \in F} d(x,y).\]
It is evident that \(f\) is continuous, hence must achieve its minimum \(m\) for some \(x \in K\).  Now if \(m = 0\), this implies that \(x \in \overline{F} = F\), hence \(K \cap F \neq \emptyset\), which is a contradiction to the given.  Thus
\[0 < m \leq \inf_{x \in K} f(x) = \inf_{x \in K, y \in F} d(x,y).\]



\item Show why the Least Upper Bound Property (every set bounded above has a least upper bound) implies the Cauchy Completeness Property (every Cauchy sequence has a limit) of the real numbers.

{\bf Solution}

Let \(\{x_n\}_{n = 1}^{\infty} \subset \mathbb{R}\) be a Cauchy sequence.  Let \(N_i\) be large enough such that \(|x_n - x_m| < 1/i\) for all \(n,m \geq N_i\), and also such that \(N_i < N_{i + 1}\), \(i \geq 1\).  Let
\[E_i = \bigcup_{n = N_i}^{\infty} \{x_n\},\]
\[\alpha_i = \sup E_i,\]
whose existence is guaranteed by the Least Upper Bound Property of \(\mathbb{R}\), since \(E_i\) is bounded above by \(x_{N_i} + 1/i\).  Then \(\{\alpha_i\}_{i = 1}^{\infty}\) is a monotonically decreasing sequence bounded below by \(x_{N_1} - 1\).  Since the Least Upper Bound Property and the Greatest Lower Bound Property are equivalent for \(\mathbb{R}\) (by the isomorphism \(x \leftrightarrow -x\)), it follows that \(\bigcup_{i = 1}^{\infty} \{\alpha_i\}\) has a greatest lower bound, say, \(\alpha^* \in \mathbb{R}\).

Now given \(\epsilon > 0\), we can choose \(i\) such that \(\alpha_i - \alpha^* < \epsilon\) and such that \(1/i < \epsilon\).  Then for \(n \geq N_i\), \(x_n \in E_i\), hence \(\alpha_i - x_n \leq 1/i < \epsilon\), hence
\[|x_n - \alpha^*| \leq |x_n - \alpha_i| + |\alpha_i - \alpha^*| < 2\epsilon,\]
showing that \(x_n \to \alpha^*\).



\item Show that there is a subset of the real numbers which is not the countable intersection of open subsets.

{\bf Solution}

\(\mathbb{Q}\) cannot be expressed as a countable intersection of open sets.

(S02.2)



\item By integrating the series
\[\frac{1}{1 + x^2} = 1 - x^2 + x^4 - x^6 + x^8 \cdots\]
prove that \(\frac{\pi}{4} = 1 - \frac{1}{3} + \frac{1}{5} - \frac{1}{7} + \frac{1}{9} \cdots\).  Justify carefully all the steps (especially taking the limit as \(x \to 1\) from below).

{\bf Solution}

We first note that
\[\int_0^{\alpha} \frac{1}{1 + x^2} dx = \arctan(\alpha) \to \frac{\pi}{4}\]
as \(\alpha \uparrow 1\).  Also, for all \(\alpha \in [0,1)\),
\[  \int_0^{\alpha} \frac{1}{1 + x^2} dx
  = \int_0^{\alpha} \left( \sum_{i = 0}^{\infty} (-1)^i x^{2i} \right) dx
  = \sum_{i = 0}^{\infty} (-1)^i \int_0^{\alpha} x^{2i} dx
  = \sum_{i = 0}^{\infty} (-1)^i \frac{\alpha^{2i + 1}}{2i + 1},\]
where the exchange of the integral and summation is justified by the uniform convergence of the series on \([0, \alpha]\).  Set
\[S_n(\alpha) = \sum_{i = 0}^n (-1)^i \frac{\alpha^{2i + 1}}{2i + 1},\]
\[S(\alpha) = \lim_{n \to \infty} S_n(\alpha)\]
for \(\alpha \in [0,1]\).  We know that \(S(\alpha) = \arctan(\alpha)\) for \(\alpha \in [0,1)\), hence we just need to show that \(S(\alpha) \to S(1)\) as \(\alpha \uparrow 1\).  Indeed, for any fixed \(n \geq 0\),
\[\begin{array}{rcl}
  |S(1) - S(\alpha)|
  & \leq & |S(1) - S_n(1)| + |S_n(1) - S_n(\alpha)| + |S_n(\alpha) - S(\alpha)| \\
  &    < & \frac{1}{2n + 3} + |S_n(1) - S_n(\alpha)| + \frac{\alpha^{2n + 3}}{2n + 3} \\
  &    < & \frac{2}{2n + 3} + |S_n(1) - S_n(\alpha)|
  \end{array}.\]
As \(\alpha \uparrow 1\), \(S_n(\alpha) \to S_n(1)\), hence
\[\lim_{\alpha \uparrow 1} |S(1) - S(\alpha)| \leq \frac{2}{2n + 3},\]
and since \(n\) was arbitrary, we conclude that
\[\lim_{\alpha \uparrow 1} S(\alpha) = S(1),\]
hence
\[1 - \frac{1}{3} + \frac{1}{5} - \frac{1}{7} + \frac{1}{9} - \cdots = S(1) = \lim_{\alpha \uparrow 1} S(\alpha) = \lim_{\alpha \uparrow 1} \arctan(\alpha) = \frac{\pi}{4}.\]



\item Suppose \(f : \mathbb{R}^2 \to \mathbb{R}\) has partial derivatives at every point bounded by \(A > 0\).

\begin{enumerate}
\item Show that there is an \(M > 0\) such that
\[|f((x,y)) - f((x_1,y_1))| \leq M((x - x_1)^2 + (y - y_1)^2)^{1/2}.\]

\item What is the smallest value of \(M\) (in terms of \(A\)) for which this always works?

\item Give an example where that value of \(M\) makes the inequality an equality.

\end{enumerate}

{\bf Solution}

\begin{enumerate}
\item Given \((x,y), (x_1,y_1) \in \mathbb{R}^2\), two applications of the Mean Value Theorem yields points \(c\) between \(x\) and \(x_1\) and \(d\) between \(y\) and \(y_1\) such that
\[f(x_1,y) - f(x,y) = \frac{\partial f}{\partial x}(c,y)(x_1 - x),\]
\[f(x_1,y_1) - f(x_1,y) = \frac{\partial f}{\partial y}(x_1,d)(y_1 - y),\]
hence
\[|f(x_1,y_1) - f(x,y)| \leq A|x_1 - x| + A|y_1 - y| \leq \sqrt{2} A \sqrt{(x_1 - x)^2 + (y_1 - y)^2}.\]

\item The above inequality is tight for general \(f\) and \((x,y), (x_1,y_1) \in \mathbb{R}^2\), i.e., \(M \geq A/\sqrt{2}\).

\item Let
\[f(x,y) = A(x + y).\]
Then
\[|f(x + t,y + t) - f(x,y)| = 2A|t| = \sqrt{2} A \sqrt{t^2 + t^2}.\]

\end{enumerate}



\item Suppose \(F : \mathbb{R}^3 \to \mathbb{R}^2\) is continuously differentiable.  Suppose for some \(v_0 \in \mathbb{R}^3\) and \(x_0 \in \mathbb{R}^2\) that \(F(v_0) = x_0\) and \(F'(v_0) : \mathbb{R}^3 \to \mathbb{R}^2\) is onto.  Show that there is a continuously differentiable function \(\gamma\), \(\gamma : (-\epsilon,\epsilon) \to \mathbb{R}^3\) for some \(\epsilon > 0\), such that
\begin{enumerate}
\item \(\gamma'(0) \neq \vec{0} \in \mathbb{R}^3\), and
\item \(F(\gamma(t)) = x_0\) for all \(t \in (-\epsilon,\epsilon)\).
\end{enumerate}

{\bf Solution}



\item Let \(T : V \to W\) be a linear transformation of finite dimensional real vector spaces.  Define the transpose of \(T\) and then prove both of the following:

\begin{enumerate}
\item \(\im(T)^{\circ} = \ker(T^t)\), where \(\im(T)^{\circ}\) is the annihilator of \(\im(T)\), the image (range of \(T\), and \(\ker(T^t)\) is the kernel (null space) of \(T^t\).

\item \(\rank(T) = \rank(T^t)\), where the rank of a linear transformation is the dimension of its image.

\end{enumerate}

{\bf Solution}

\(T^t : W^* \to V^*\) is defined such that \(T^t(g) = g \circ T \in V^*\) for \(g \in W^*\).

(F01.7)



\item Let \(T\) be the rotation of an angle \(60^{\circ}\) counterclockwise about the origin in the plane perpendicular to \((1,1,2)\) in \(\mathbb{R}^3\).

\begin{enumerate}
\item Find the matrix representation of \(T\) in the standard basis.  Find all eigenvalues and eigenspaces of \(T\).

\item What are the eigenvalues and eigenspaces of \(T\) if \(\mathbb{R}^3\) is replaced by \(\mathbb{C}^3\).

\end{enumerate}

(You do not have to multiply any matrices out but must compute any inverses.)

{\bf Solution}

\begin{enumerate}
\item Let
\[B_T = \matrixiiibyiii{\frac{1}{\sqrt{6}}}{ \frac{1}{\sqrt{2}}}{ \frac{1}{\sqrt{3}}}
                       {\frac{1}{\sqrt{6}}}{-\frac{1}{\sqrt{2}}}{ \frac{1}{\sqrt{3}}}
                       {\frac{2}{\sqrt{6}}}{                  0}{-\frac{1}{\sqrt{3}}}.\]
Then regarding the columns of \(B_T\) as an orthonormal basis, the matrix representation of \(T\) in this basis is
\[[T]_{B_T} = \matrixiiibyiii{1}{               0}{              0}
                             {0}{ \cos 60^{\circ}}{\sin 60^{\circ}}
                             {0}{-\sin 60^{\circ}}{\cos 60^{\circ}},\]
so the matrix representation of \(T\) in the standard basis is
\[T = B_T [T]_{B_T} B_T^{-1}
    = B_T [T]_{B_T} B_T^t.\]
The only eigenvalue of \(T\) is \(1\), with corresponding eigenspace \(\span \{(1 \ 1 \ 2)\}\).

\item If we consider complex eigenvalues and eigenspaces, then \(T\) additionally has eigenvalues \(e^{i 60^{\circ}}\) and \(e^{-i 60^{\circ}}\) with corresponding eigenspaces \(\span \{(1 \ i)\}\) and \(\span \{(1 \ -i)\}\), respectively.

\end{enumerate}



\item Let \(V\) be a complex inner product space.  State and prove the Cauchy-Schwarz inequality.

{\bf Solution}

Given \(u,v \in V\), the Cauchy-Schwarz inequality states that
\[|(u,v)| \leq \|u\| \|v\|,\]
with equality if and only if \(u\) and \(v\) are linearly dependent.

To prove the claim, consider
\[0 \geq (u - tv, u - tv) = \|u\|^2 + |t|^2 \|v\|^2 - t\overline{(u,v)} - \overline{t}(u,v)\]
for any \(t \in \mathbb{C}\).  Now if \(v = 0\), the claim is trivial, so suppose \(v \neq 0\) and set \(t = (u,v) / \|v\|^2\) to yield
\[0 \geq \|u\|^2 + \frac{|(u,v)|^2}{\|v\|^2} - \frac{|(u,v)|^2}{\|v\|^2} - \frac{|(u,v)|^2}{\|v\|^2}
       = \|u\|^2 - \frac{|(u,v)|^2}{\|v\|^2},\]
from which the first part of the claim follows immediately.  Clearly, equality holds if and only if \(u - tv = 0\) for some value of \(t\).



\item Let \(A\) be an \(n \times n\) complex matrix satisfying \(A^*A = AA^*\), where \(A^*\) is the adjoint of \(A\).  Let \(V = \mathbb{C}^{\{n \times 1\}}\), the \(n \times 1\) complex column matrices, be an inner product space under the dot product.  View \(A : V \to V\) as a linear map.  Prove that there exists an orthonormal basis of \(V\) consisting of eigenvectors of \(A\), i.e., prove this form of the Spectral Theorem for normal operators.

{\bf Solution}

Let \(\lambda\) be an eigenvalue of \(A\) (whose existence is given by the roots of the characteristic polynomial), and let \(E_{\lambda} = \{x \in V \ | \ A x = \lambda x\}\) be the corresponding eigenspace.  Then given any \(x \in E_{\lambda}\),
\[A(A^*x) = A^*(Ax) = A^*(\lambda x) = \lambda (A^*x),\]
thus \(A^*x \in E_{\lambda}\) and \(A^*\) is invariant on \(E_{\lambda}\).  Now given any \(y \in E_{\lambda}^{\perp}\), if \(x \in E_{\lambda}\), then
\[(Ay,x) = (y,A^*x) = 0,\]
since \(y \in E_{\lambda}^{\perp}\) and \(A^*x \in E_{\lambda}\).  Thus \(Ay \in E_{\lambda}^{\perp}\) and \(A\) is invariant on \(E_{\lambda}^{\perp}\).  It follows that the restriction of \(A\) to \(E_{\lambda}^{\perp}\) is a normal linear transformation, hence, by induction, has an orthonormal basis consisting of eigenvectors of \(A\).  Combine this orthonormal basis with an orthonormal basis for \(E_{\lambda}\) and we arrive at the desired conclusion.



\end{enumerate}

\end{document}
