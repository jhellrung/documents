\documentclass{article}

%\usepackage[left=1in,top=1in,bottom=1in,right=1in,nohead,nofoot]{geometry}
\usepackage{fullpage}
\usepackage{amsmath}
\usepackage{amsfonts}
\usepackage{graphicx}


\def\tr{\mathop{\rm tr}\nolimits}
\def\ker{\mathop{\rm ker}\nolimits}
\def\dim{\mathop{\rm dim}\nolimits}
\def\im{\mathop{\rm im}\nolimits}
\def\span{\mathop{\rm span}\nolimits}
\def\dist{\mathop{\rm dist}\nolimits}


\begin{document}


\begin{flushright}
Jeffrey Hellrung \\
Basic Examination, S05 \\
\end{flushright}


\begin{enumerate}

\item Given \(n \geq 1\), let \(\tr : M_n(\mathbb{C}) \to \mathbb{C}\) denote the trace of a matrix:
\[\tr(A) = \sum_{k = 1}^n A_{k,k}.\]

\begin{enumerate}
\item Determine a basis for the kernel (or null-space) of \(\tr\).

\item For \(X \in M_n(\mathbb{C})\), show that \(\tr(X) = 0\) if and only if there exists an integer \(m\) and matrices \(A_1, \ldots, A_m, B_1, \ldots, B_m \in M_n(\mathbb{C})\) so that
\[X = \sum_{j = 1}^m A_j B_j - B_j A_j.\]

\end{enumerate}

{\bf Solution}

\begin{enumerate}
\item Let \(E_{ij}\) be the matrix with \(ij\) entry equal to \(1\) and the rest \(0\).  Then a basis for the kernel of \(\tr\) includes \(E_{ij}\) for \(i \neq j\) and \(E_{11} - E_{ii}\) for \(i > 1\).  Note that this is a linearly independent set of matrices, and they number \((n^2 - n) + (n - 1) = n^2 - 1\), which is the dimension of \(\ker(\tr)\) (since \(\dim(\im(\tr)) = 1\) and \(\dim(M_n(\mathbb{C})) = n^2\)), so must span \(\ker(\tr)\).

\item Note that
\[E_{ij} E_{k\ell} - E_{k\ell} E_{ij} = E_{i\ell} \delta_{jk} - E_{kj} \delta_{\ell i}.\]
Thus for \(\ell \neq i\),
\[E_{ij} E_{j\ell} - E_{j\ell} E_{ij} = E_{i\ell},\]
and for \(i > 1\),
\[E_{i1} E_{1i} - E_{1i} E_{i1} = E_{ii} - E_{11}.\]
It follows that the set of matrices \(\{E_{ij} E_{k\ell} - E_{k\ell} E_{ij}\}\) spans \(\ker(\tr)\), from which we can deduce the given representation.

\end{enumerate}



\item  Let \(V\) be a finite-dimensional vector space, and let \(V^*\) denote the dual space; that is, the space of linear maps \(\phi : V \to \mathbb{C}\).  For a set \(W \subset V\), let
\[W^{\perp} = \{\phi \in V^* : \phi(w) = 0 \ \forall w \in W\}.\]
For a subset \(U \subset V^*\), let
\[^{\perp}U = \{v \in V : \phi(v) = 0 \ \forall \phi \in U\}.\]

\begin{enumerate}
\item Show that for any subset \(W \subset V\), \(^{\perp}(W^{\perp}) = \span(W)\).

Recall that the span of a set of vectors is the smallest vector subspace that contains these vectors.

\item Let \(W \subset V\) be a linear subspace.  Give an explicit isomorphism between \((V/W)^*\) and \(W^{\perp}\).  Show that it is an isomorphism.

\end{enumerate}

{\bf Solution}

\begin{enumerate}
\item We first show that \(W^{\perp} = \span(W)^{\perp}\).  Indeed, given any \(\phi \in W^{\perp}\), any \(w \in \span(W)\) can be expressed as a linear combination of vectors in \(W\), hence \(\phi(w) = 0\) and \(\phi \in \span(W)^{\perp}\).  Conversely, \(W \subset \span(W)\), so certainly any \(\phi\) vanishing on \(\span(W)\) will likewise vanish on \(W\).  This shows the claim.  We are thus reduced to the case of showing \(^{\perp}(W^{\perp}) = W\) for a subspace \(W\) of \(V\).

Let \(\{w_1, \ldots, w_k\}\) be a basis for \(W\), and complete this with \(\{v_{k + 1}, \ldots, v_n\}\) to give a basis for \(V\).  Then \(\{v_i^*\}_{i = k + 1}^n\) is a basis for \(W^{\perp}\).  Certainly \(w \in W\) if and only if \(v_i^*(w) = 0\) for \(i = k + 1, \ldots, n\), hence \(^{\perp}(W^{\perp}) = W\), as claimed.

\item Let \(\phi \in W^{\perp}\); then define \(\Phi \in (V/W)^*\) by
\[\Phi \{v\} = \phi(v).\]
It follows from \(\phi \in W^{\perp}\) that this is well-defined, hence provides an injection from \(W^{\perp}\) to \((V/W)^*\).  Conversely, given \(\Phi \in (V/W)^*\), we can define \(\phi \in V^*\) by
\[\phi(v) = \Phi \{v\},\]
and since \(\Phi \{v\} = \Phi \{0\} = 0\) for any \(v \in W\), we have that \(\phi \in W^{\perp}\).  This provides an injection in the other direction, hence the two spaces are isomorphic.

\end{enumerate}



\item Let \(A\) be a Hermitian-symmetric \(n \times n\) complex matrix.  Show that if \((Av,v) \geq 0\) for all \(v \in \mathbb{C}^n\), then there exists an \(n \times n\) matrix \(T\) so that \(A = T^*T\).

{\bf Solution}



\item Let \(\mathcal{A} = M_n(\mathbb{C})\) denote the set of all \(n \times n\) matrices with complex entries.

We say that \(\mathcal{I} \subset \mathcal{A}\) is a {\em two-sided ideal} in \(\mathcal{A}\) if
\begin{enumerate}
\item for all \(A,B \in \mathcal{I}\), \(A + B \in \mathcal{I}\)
\item for all \(A \in \mathcal{I}\) and \(B \in \mathcal{A}\), \(AB\) and \(BA\) belong to \(\mathcal{I}\)
\end{enumerate}
Show that the only two-side ideals in \(\mathcal{A}\) are \(\{0\}\) and \(\mathcal{A}\) itself.

{\bf Solution}

Suppose \(\mathcal{I} \neq \{0\}\).  Then there exists some \(A \in \mathcal{I}\) with \(A \neq 0\).  Let \(\alpha = (A)_{rc} \neq 0\) be the \(rc^{th}\) entry of \(A\).

Let \(E_{ij} \in \mathcal{A}\) have \(ij^{th}\) entry equal to \(1\) and the rest \(0\).  Then by iterative applications of (b),
\[E_{ir}AE_{cj} \left( \frac{c}{\alpha}I \right) = cE_{ij} \in \mathcal{I}\]
for any \(i,j\) and \(c \in \mathbb{C}\).  It follows from (a) and the fact that \(\span \{E_{ij}\} = \mathcal{A}\) that \(\mathcal{I} = \mathcal{A}\).



\item For a subset \(X \subset \mathbb{R}\), we say that \(X\) is {\em algebraic}, if there exists a family \(\mathcal{F}\) of polynomials with rational coefficients, so that \(x \in X\) if and only if \(p(x) = 0\) for some \(p \in \mathcal{F}\).

\begin{enumerate}
\item Show that the set \(\mathbb{Q}\) of rational numbers is algebraic.

\item Show that the set \(\mathbb{R} \backslash \mathbb{Q}\) of irrational numbers is not algebraic.

\end{enumerate}

{\bf Solution}

\begin{enumerate}
\item Let \(\mathcal{F} = \{x \mapsto qx - p \ | \ q,p \in \mathbb{Z}\}\).  Then for each \(x = \frac{p}{q} \in \mathbb{Q}\), with \(p,q \in \mathbb{Z}\), \(q \neq 0\), \(x\) is a root of \(qx - p\), so it follows that \(\mathbb{Q}\) is algebraic.

\item The set of all polynomials \(P\) with rational coefficients is countable.  Indeed, if
\[P_n = \left\{ \left. x \mapsto \sum_{i = 0}^n a_i x^i \ \right| \ a_i \in \mathbb{Q} \right\}\]
is the set of all polynomials of degree \(n\) or less, we see that each \(p \in P_n\) can be associated with the \(n\)-tuple of its coefficients, hence \(P_n \cong \mathbb{Q}^n\), which is countable.

Now each \(p \in P_n\) has at most \(n\) roots, hence the set of \(x \in \mathbb{R}\) for which \(x\) is a root of some \(p \in P_n\) is countable for each \(n\), and it follows that the set of all algebraic numbers themselves, which is the union of all such \(x\) over \(n\), is countable.  Since \(\mathbb{R} \backslash \mathbb{Q}\) is uncountable, it therefore can't be algebraic.

\end{enumerate}



\item Let \(X\) be the set of all infinite sequences \(\{\sigma_n\}_{n = 1}^{\infty}\) of \(1\)'s and \(0\)'s endowed with the metric
\[\dist \left( \{\sigma_n\}_{n = 1}^{\infty},
               \{\sigma'_n\}_{n = 1}^{\infty} \right)
  = \sum_{n = 1}^{\infty} \frac{1}{2^n} \left| \sigma_n - \sigma'_n \right|.\]
Give a direct proof that every infinite subset of \(X\) has an accumulation point.

{\bf Solution}

Let \(S\) be an infinite subset of \(X\).  Construct nested infinite subsets \(S_0, S_1, S_2, \ldots\) of \(S\) as follows.  Set \(S_0 = S\), and given \(S_n\) infinite, select \(\sigma_{n + 1}\) such that infinitely many of the sequences in \(S_n\) have \((n + 1)^{th}\) term equal to \(\sigma_{n + 1}\).  Then set \(S_{n + 1}\) to be the set of sequences of \(S_n\) with \((n + 1)^{th}\) term equal to \(\sigma_{n + 1}\).  Thus \(S_{n + 1}\) is also infinite, and the process can continue.

The claim is that the sequence \(s = \{\sigma_i\}_{i = 1}^{\infty}\) is an accumulation point of \(S\).  Indeed, any \(s' = \{\sigma'_i\}_{i = 1}^{\infty} \in S_n\) agrees with \(s\) in the first \(n\) terms, hence
\[\dist(s, s')
     = \sum_{i = n + 1}^{\infty} \frac{1}{2^i} \left| \sigma_i - \sigma'_i \right|
  \leq \sum_{i = n + 1}^{\infty} \frac{1}{2^i}
     = \frac{1}{2^n}\]
which tends to \(0\) as \(n \to \infty\).  Since each \(S_n\) has infinitely many sequences, it follows that \(s\) is an accumulation point of \(S\).



\item Let \(X,Y\) be two topological spaces.  We say that a continuous function \(f : X \to Y\) is {\em proper} if \(f^{-1}(K)\) is compact for any compact set \(K \subset Y\).

\begin{enumerate}
\item Give an example of a function that is proper but not a homeomorphism.

\item Give an example of a function that is continuous but not proper.

\item Suppose \(f : \mathbb{R} \to \mathbb{R}\) is \(C^1\) (that is, has a continuous derivative) and
\[|f'(x)| \geq 1 \ \text{ for all } x \in \mathbb{R}.\]
Show that \(f\) is proper.

\end{enumerate}

{\bf Solution}

\begin{enumerate}
\item Let \(f : \mathbb{R} \to [0,\infty)\) be the function \(x \mapsto |x|\).  Then if \(K \subset [0,\infty)\) is compact, \(f^{-1}(K) = K \cup -K\) is certainly compact, where \(-K = \{x \ | \ -x \in K\}\), hence \(f\) is proper, but \(f\) is not injective hence not a homeomorphism.

\item Let \(f :\mathbb{R} \to \{0\}\) be the function \(x \mapsto 0\).  Then \(f^{-1}(\{0\}) = \mathbb{R}\), which is not compact, hence \(f\) is not proper.

\item We first show that \(f\) is a homeomorphism, i.e., bijective.  Let \(x,y \in \mathbb{R}\).  Then by the Mean Value Theorem,
\[|f(x) - f(y)| = |f'(c)| |x - y| \geq |x - y|\]
for some \(c\) between \(x\) and \(y\), which shows that \(f\) is injective.  Now \(f'\) is continuous, hence either \(f' \geq 1\) or \(f' \leq -1\) on all of \(\mathbb{R}\).  If \(f' \geq 1\), we have
\[f(y) - f(0) \geq y\]
for any \(y \geq 0\), so, since \(f\) is continuous, by the Intermediate Value Theorem there exists an \(x \in [0,y]\) such that \(f(x) = f(0) + y\).  A similar argument can be made for \(y \leq 0\) and for when \(f' \leq -1\), and we conclude that \(f\) is surjective, hence a homeomorphism.

Now let \(K \subset \mathbb{R}\) be compact.  Then \(K\) is closed, hence \(f^{-1}(K)\) is closed by the continuity of \(f\).  Further, \(K\) is bounded, say, \(|y| < M\) for all \(y \in K\).  Then \(f^{-1}(-M)\) and \(f^{-1}(M)\) bound \(f^{-1}(K)\), for take any \(x \in f^{-1}(K)\) and set \(y = f(x)\).  Then \(y\) is bounded between \(-M\) and \(M\), hence \(x\) is bounded between \(f^{-1}(-M)\) and \(f^{-1}(M)\) (either \(f\) or \(-f\) preserves ordering).  Thus \(f^{-1}(K)\) is closed and bounded, hence compact, and it follows that \(f\) is proper.

\end{enumerate}



\item Suppose \(f : \mathbb{R} \to \mathbb{R}\) is \(C^1\) (i.e., continuously differentiable).  Show that
\[\lim_{n \to \infty} \sum_{j = 1}^n \left| f \left( \frac{j - 1}{n} \right) -
                                            f \left( \frac{j}{n} \right) \right|\]
is equal to
\[\int_0^1 |f'(t)| dt.\]

{\bf Solution}

\(f'\) is continuous on \([0,1]\), a compact set, hence uniformly continuous.  Let \(\epsilon > 0\) be given.  Then there exists an \(N\) such that, if \(n > N\), \(|f'(x) - f'(y)| < \epsilon\) whenever \(|x - y| < \frac{1}{n}\).  By the Mean Value Theorem,
\[f \left( \frac{j}{n} \right) - f \left( \frac{j - 1}{n} \right)
  = f'(x_j) \frac{1}{n},\]
\(x_j \in \left( \frac{j}{n}, \frac{j - 1}{n} \right)\), for each \(j = 1, \ldots, n\).  Let
\[P = \left\{ \frac{j}{n} \right\}_{j = 0}^{n}\]
be a partition of \([0,1]\), and set
\[M_j = \sup_{\left[ \frac{j - 1}{n}, \frac{j}{n} \right]} |f'|,\]
\[m_j = \inf_{\left[ \frac{j - 1}{n}, \frac{j}{n} \right]} |f'|\]
for \(j = 1, \ldots, n\).  Then
\[M_j - |f'(x_j)| < \epsilon,\]
\[|f'(x_j)| - m_j < \epsilon,\]
thus
\[U(P,|f'|) - \sum_{j = 1}^n \left| f \left( \frac{j - 1}{n} \right) -
                                 f \left( \frac{j}{n} \right) \right|
  < \sum_{j = 1}^n \epsilon \frac{1}{n}
  = \epsilon,\]
and similarly
\[\sum_{j = 1}^n \left| f \left( \frac{j - 1}{n} \right) -
                        f \left( \frac{j}{n} \right) \right| - L(P,|f'|)
  < \epsilon,\]
hence
\[U(P,|f'|) - L(P,|f'|) < 2\epsilon,\]
so \(|f'|\) is Riemann integrable on \([0,1]\).  Further,
\[\left| \int_0^1 |f'(t)| dt
  - \sum_{j = 1}^n \left| f \left( \frac{j - 1}{n} \right) -
                          f \left( \frac{j}{n} \right) \right|
  \right| < \epsilon\]
for all \(n > N\), which proves the claim.



\item

\begin{enumerate}
\item Suppose
\[\lim_{n \to \infty} a_n = A.\]
Show that
\[\lim_{N \to \infty} \frac{1}{N} \sum_{n = 1}^N a_n = A.\]

\item Show by example that the converse is false.

\end{enumerate}

{\bf Solution}

(F04.1)



\item Consider the set of \(f : [0,1] \to \mathbb{R}\) that obey
\[|f(x) - f(y)| \leq |x - y| \ \text{ and } \ \int_0^1 f(x) dx = 1.\]
Show that this is a compact subset of \(C([0,1])\).

{\bf Solution}

By the Arzela-Ascoli Theorem, \(A \subset C([0,1])\) is compact if and only if
\begin{enumerate}
\item \(A\) is closed.
\item \(A\) is uniformly bounded.
\item \(A\) is equicontinuous.
\end{enumerate}
Let \(A\) be the subset described above.  To show (a), suppose \(\{f_n\}_{n = 1}^{\infty}\) is a Cauchy sequence of functions in \(A\).  Then, for each \(x \in [0,1]\), \(\{f_n(x)\}_{n = 1}^{\infty}\) is a Cauchy sequence in \(\mathbb{R}\), hence converges.  Define \(f : [0,1] \to \mathbb{R}\) by \(f(x) = \lim_{n \to \infty} f_n(x)\).  Then given \(\epsilon > 0\), there exists an \(N\) such that \(\sup_{[0,1]} |f_n - f_m| < \epsilon\) for \(n,m > N\), hence, letting \(m \to \infty\), we see that \(\sup_{[0,1]} |f_n - f| \leq \epsilon\) for \(n > N\).  Thus
\[|f(x) - f(y)| \leq |f_n(x) - f_n(y)| + 2\epsilon \leq |x - y| + 2\epsilon\]
for any \(x,y \in [0,1]\), and since \(\epsilon\) was arbitrary, \(f\) satisfies the first condition of being in \(A\).  Note that \(f_n \to f\) uniformly on \([0,1]\), hence
\[\int_0^1 f dx = \lim_{n \to \infty} \int_0^1 f_n dx = 1,\]
and \(f\) satisfies the second condition of being in \(A\).  Therefore, \(f \in A\) and \(A\) is closed.

To show (b), let \(f \in A\).  Then the first condition implies that
\[|f(x) - f(0)| \leq x \leq 1\]
for all \(x \in [0,1]\), hence
\[f(0) - 1 \leq f(x) \leq f(0) + 1.\]
Now using the second condition,
\[1 = \int_0^1 f(x) dx \geq f(0) - 1,\]
so \(f(0) \leq 2\) and \(f \leq 3\) on \([0,1]\).  On the other hand,
\[1 = \int_0^1 f(x) dx \leq f(0) + 1,\]
so \(f(0) \geq 0\) and \(f \geq -1\) on \([0,1]\).  Thus \(f\) is bounded in absolute value by \(3\), and since \(f\) was arbitrary, \(A\) is uniformly bounded.

(c) follows directly from the first condition, so we conclude that \(A\) is compact.



\item Let us make \(M_n(\mathbb{C})\) into a metric space in the following fashion:
\[\dist(A,B) = \left( \sum_{i,j} \left| A_{ij} - B_{ij} \right|^2 \right)^{1/2}\]
(which is just the usual metric on \(\mathbb{R}^{n^2}\)).

\begin{enumerate}
\item Suppose \(F : \mathbb{R} \to M_n(\mathbb{C})\) is continuous.  Show that the set
\[\left\{ x \in \mathbb{R} : F(x) \text{ is invertible} \right\}\]
is open (in the usual topology on \(\mathbb{R}\)).

\item Show that on the set given above, \(x \mapsto F(x)^{-1}\) is continuous.

\end{enumerate}

{\bf Solution}

\begin{enumerate}
\item Since \(\det A\) is a continuous function of the entries of \(A\) (indeed, it is a polynomial in the entries of \(A\)), \(x \mapsto \det F(x)\) is a continuous mapping, hence the inverse image of \(\mathbb{R} \backslash \{0\}\) under this mapping is open, which is precisely those \(x\) such that \(F(x)\) is invertible.

\item Since the entries of \(A^{-1}\) are continuous functions of the entries of \(A\) (indeed, rational functions of the entries of \(A\)), it follows that \(x \mapsto F(x)^{-1}\) is continuous.

\end{enumerate}



\item Let \((X,d)\) be a metric space.  Prove that the following are equivalent:
\begin{enumerate}
\item There is a countable dense set.
\item There is a countable basis for the topology.
\end{enumerate}
Recall that a collection of open sets \(\mathcal{U}\) is called a basis if every open set can be written as a union of elements of \(\mathcal{U}\).

{\bf Solution}

Let \(Y\) be a countable dense set of \(X\).  Consider the family of open sets
\[\mathcal{U} = \left\{ B_d(y;r) \ | \ y \in Y, r \in \mathbb{Q} \right\}.\]
Note that \(\mathcal{U}\) is countable.  Further, given some open set \(U \subset X\), every \(x \in U\) is covered by some \(B_d(y_x;r_x) \subset U\) (since \(Y\) is dense and \(U\) is open), hence
\[U = \bigcup_{x \in U} B_d(y_x;r_x),\]
from which it follows that \(\mathcal{U}\) is a countable basis for \((X,d)\).

Conversely, let \(\mathcal{U} = \{U_n\}_{n = 1}^{\infty}\) be a countable basis for \((X,d)\), and choose \(x_n \in U_n\).  Then since every \(B(x;r)\) contains some \(U_n\) for every \(x \in X\), \(r \in \mathbb{R}\), \(\{x_n\}_{n = 1}^{\infty}\) is a countable dense subset of \(X\).



\end{enumerate}

\end{document}
