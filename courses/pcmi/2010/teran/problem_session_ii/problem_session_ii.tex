\documentclass{article}

\usepackage{fullpage}
\usepackage{amsmath}
\usepackage{amssymb}
\usepackage{amsthm}
\usepackage{graphicx}

\title{Simluation of Elasticity, Biomechanics, and Virtual Surgery \\
       Problem Session II}
\date{July 14, 2010}
\author{Joseph Teran (UCLA) \and
        Jeffrey Hellrung (UCLA)}

\begin{document}

\maketitle

\section{Elasticity}

The goal in this problem session is to use a (variable-coefficient) Poisson solver (either one you coded from Problem Session I or the one we provide) to implement the Neo-Hookean model of elasticity in dimension \(d = 1\).  Recall that the equations of elasticity are given by the boundary value problem

\begin{subequations}\label{elasticity}
\begin{align}
\rho_0(\mathbf{X}) \frac{\partial^2 \mathbf{u}}{\partial t^2} & = \nabla^{\mathbf{X}} \cdot \mathbf{P} + \mathbf{f}^{\text{ext}} \quad \in \Omega(t) \\
\mathbf{u}(\cdot,t) & = \mathbf{g}(\cdot,t) \quad \in \partial \Omega_d(t) \\
\left( \mathbf{P} \cdot \mathbf{\hat{n}} \right) (\cdot,t) & = \mathbf{h}(\cdot,t) \quad \in \partial \Omega_n(t)
\end{align}
\end{subequations}

where \(\rho_0\) is the mass density (as a function of \(\mathbf{X}\), the undeformed coordinates); \(\mathbf{u} = \mathbf{\phi} - \mathbf{X}\) is the unknown displacement; \(\mathbf{P}\) is the first Piola-Kirchoff stress (which takes a specific form for Neo-Hookean, to be given later); \(\mathbf{f}^{\text{ext}}\) is the given (external) force; and \(\mathbf{g}\) and \(\mathbf{h}\) specify the Dirichlet and Neumann boundary conditions, respectively.  For simplicity, we'll assume a uniform mass density, i.e., \(\rho_0 \equiv 1\).

\section{Neo-Hookean}

The Neo-Hookean model relates the stress \(\mathbf{P}\) to the deformation gradient \(\mathbf{F}\) via

\begin{subequations}
\begin{align}
\Psi(\mathbf{F}) & = \frac{\mu}{2} \left( F_{ij} F_{ji} - 2 \right) - \mu \log J + \frac{\lambda}{2} \log^2 J, \label{neohookeanstrainenergydensity} \\
\mathbf{P}(\mathbf{F}) = \frac{\partial \Psi}{\partial \mathbf{F}} & = \mu \mathbf{F} + \left( \lambda \log J - \mu \right) \mathbf{F}^{-T}.
\end{align}
\end{subequations}

(Recall that \(J = \det \mathbf{F}\).)  This manifests itself in dimension \(d = 1\) in terms of the displacement \(u\) as

\begin{equation}
P(u) = \mu \left( \frac{du}{dX} + 1 \right) + \left( \lambda \log \left( \frac{du}{dX} + 1 \right) - \mu \right) \frac{1}{\frac{du}{dX} + 1}.
\end{equation}

\section{Inversion-Robust Neo-Hookean}

Neo-Hookean as formulated above will not be robust to element inversions, due to the \(\log J\) term.  To remedy this, we replace the logarithm in \eqref{neohookeanstrainenergydensity} with a cubic Taylor approximation around \(1\):

\begin{subequations}\label{r}
\begin{align}
\begin{split}
r(x) & = (x - 1) - \frac{1}{2} (x - 1)^2 + \frac{1}{3} (x - 1)^3 \\
     & = -\frac{11}{6} + 3 x - \frac{3}{2} x^2 + \frac{1}{3} x^3,
\end{split} \\
\begin{split}
r'(x) & = 1 - (x - 1) + (x - 1)^2 \\
      & = 3 - 3 x + x^2.
\end{split} \\
\begin{split}
r''(x) & = -1 + 2 (x - 1) \\
       & = -3 + 2 x.
\end{split}
\end{align}
\end{subequations}

This gives

\begin{subequations}
\begin{align}
\Psi(\mathbf{F}) & = \frac{\mu}{2} \left( F_{ij} F_{ji} - 2 \right) - \mu r(J) + \frac{\lambda}{2} r(J)^2, \\ \mathbf{P}(\mathbf{F}) = \frac{\partial \Psi}{\partial \mathbf{F}} & = \mu \mathbf{F} + \left( \lambda r(J) - \mu \right) r'(J) \frac{\partial J}{\partial \mathbf{F}}.
\end{align}
\end{subequations}

Again, specializing this for \(d = 1\) dimension, we obtain

\begin{equation}\label{P1d}
P(u) = \mu \left( \frac{du}{dX} + 1 \right) + \left( \lambda r \left( \frac{du}{dX} + 1 \right) - \mu \right) r' \left( \frac{du}{dX} + 1 \right).
\end{equation}

\section{Quasistatics}

We begin by studying eqs \eqref{elasticity} at equilibrium, which is the basis for quasistatic evolution.  This reduces the equations to

\begin{equation}\label{quasistatics}
-\nabla^{\mathbf{X}} \cdot \mathbf{P} = \mathbf{f}^{\text{ext}}.
\end{equation}

The equivalent weak formulation is

\begin{equation}
\int_{\Omega_0} w_{i,j} P_{ij} d\mathbf{X} = \int_{\partial \Omega_n} w_i h_i dS(\mathbf{X}) + \int_{\Omega_0} w_i f_i^{\text{ext}} d\mathbf{X}
\end{equation}

for all test functions \(\mathbf{w}\).  Recall that summation over repeated indices is implied, and comma'ed indices indicate differentiation.

As for Poisson, we let the coordinates of \(\mathbf{w}\) vary over the nodal basis functions \(N_i\).  In dimension \(d = 1\), this reduces to the system of equations

\begin{subequations}\label{weak1d}
\begin{align}
q_i(u) & = \int_a^b \frac{\partial N_i}{\partial X} P(F(u(X))) dX - b_i = 0; \\
b_i & = \int_a^b N_i f^{\text{ext}} dX + \text{[Neumann boundary terms]}
\end{align}
\end{subequations}

for each grid vertex \(i\).  For Neo-Hookean, \(P\) depends {\em non-linearly} on the displacement \(u\), hence one must use a non-linear solver, such as Newton iteration, to solve \eqref{weak1d} for \(u\).  The Newton step looks like

\begin{subequations}
\begin{align}
\frac{\partial q_i}{\partial u} \left( u^k \right) \Delta u + q_i \left( u^k \right) = 0; \\
u^{k+1} = u^k + \Delta u
\end{align}
\end{subequations}

where

\begin{equation}
\frac{\partial q_i}{\partial u_j} (u) = \int_a^b \frac{\partial N_i}{\partial X} \frac{\partial P}{\partial F} \left( F(u) \right) \frac{\partial N_j}{\partial X} dX
\end{equation}

Thus, the computation of \(\Delta u\) in each Newton iteration amounts to solving a variable coefficient Poisson problem.  This coefficient is \(\partial P / \partial F = \partial P / \partial \left( du/dX \right)\), which we can express via \eqref{P1d} as

\begin{equation}
\frac{\partial P}{\partial F} = \mu + \lambda r' \left( \frac{du}{dX} + 1 \right)^2 + \left( \lambda r \left( \frac{du}{dX} + 1 \right) - \mu \right) r'' \left( \frac{du}{dX} + 1 \right).
\end{equation}

This gives all the necessary pieces to implement a quasistatics evolution of the elasticity equations \eqref{elasticity}.

\section{Backward Euler}

In the event that inertial terms are non-negligible, we must use some time-stepping scheme to solve \eqref{elasticity}.  Here, we outline the implementation of Backward Euler.  Since \eqref{elasticity} is second-order in time, we must introduce an additional variable, \(\mathbf{v} = \partial \mathbf{u} / \partial t\), to convert \eqref{elasticity} to an equivalent first-order system:

\begin{equation}
\begin{split}
\frac{\partial \mathbf{u}}{\partial t} & = \mathbf{v}, \\
\rho_0 \frac{\partial \mathbf{v}}{\partial t} & = \nabla^{\mathbf{X}} \cdot \mathbf{P} + \mathbf{f}^{\text{ext}}.
\end{split}
\end{equation}

The time discretization for Backward Euler thus gives:

\begin{equation}
\begin{split}
\frac{\mathbf{u}^{m+1} - \mathbf{u}^m}{\Delta t} & = \mathbf{v}^{m+1}, \\
\rho_0 \frac{\mathbf{v}^{m+1} - \mathbf{v}^m}{\Delta t} & = \left( \nabla^{\mathbf{X}} \cdot \mathbf{P} \right)_{\mathbf{u}^{m+1}} + \mathbf{f}^{\text{ext}}.
\end{split}
\end{equation}

To solve these equations, we use the first to substitute for \(\mathbf{v}^{m+1}\) in the second and rearrange terms to obtain (again, assuming \(\rho \equiv 1\) for simplicity)

\begin{equation}
\mathbf{u}^{m+1} - \Delta t^2 \left( \nabla^{\mathbf{X}} \cdot \mathbf{P} \right)_{\mathbf{u}^{m+1}} = \Delta t^2 \mathbf{f}^{\text{ext}} + \Delta t \mathbf{v}^m + \mathbf{u}^m
\end{equation}

which turns out to be very similar to \eqref{quasistatics}.  Indeed, the difference between the two systems essentially amounts to the addition of an identity matrix.

\end{document}
